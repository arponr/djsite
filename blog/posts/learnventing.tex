Learnventing
----------------------------------------------------------------------
learnventing
----------------------------------------------------------------------
learning, math
----------------------------------------------------------------------
Today I was convinced that I really need to sit down, screw around,
and reinvent the fundamentals for myself to have any chance at
understanding or retaining math.
----------------------------------------------------------------------
This semester I've been taking a course on algebraic topology, which
is purportedly a continuation of the introduction to the subject I
took last semester. Really it's this insane plunge into some
neat(-looking) but ridiculously abstract homotopy theory. I really
hope to understand it one day. Hopefully this absurd month (four final
projects/papers!) will pass by relatively successfully and without
claiming too much of my sanity; then over the summer I can write a bit
about (and finally learn) some of the ideas that appeared in the
course. 

But today I was convinced that there is something far more important I
need to do this summer: sit down, screw around, and reinvent all the
fundamentals (which I've forgotten) for myself. Indeed that whole
first paragraph was just for the sake of
contradiction/contrast. Here's the actual point of the post.

I'm also taking a course on number fields this semseter, and section
today was comprised of just our TF, the infamous [Levent Alpoge](http://people.fas.harvard.edu/~alpoge/), and
me. (My current understanding: Levent is essentially

1. an unrelenting and rather talented troll,
2. a total boss.

I'm not sure if I got the order there right, but I'll leave it like
that in the hopes that he reads this.) Anyway, Levent asks me if I
want to talk about the theorem on primes in arithmetic progressions
due to Dirichlet. I say yes, and that I know nothing of how the proof
goes. He's about to start talking---he says often in number theory we
have to do some algebra and some analysis, and usually the analysis is
where the deep, tricky stuff is happening; I'm only writing that here
because I want to remember that---but, true to form, instead
of actually starting the proof, he decides that today I am Dirichlet
and I will prove this theorem. 

Obviously he was exaggerating and presented the bulk of the argument
(that is, the analysis), but I had to try and figure out some of the
algebra. Basically what I ended up doing is realising that, if $p \in
\mathbb{N}$ is prime and $a, n \in \mathbb{N}$ with $(a,n) = 1$, one
can write the indicator function
\[
p \mapsto 
\begin{cases}
1 & p \equiv a \pod{n}, \\
0 & \text{otherwise}
\end{cases}
\]
algebraically as
\[
p \mapsto \frac{1}{\varphi(n)} \sum_{\chi \in
  ((\mathbb{Z}/n)^\times)^\vee} \chi(p)\overline{\chi(a)},
\]
where $G^\vee := \mathrm{Hom}_{\mathrm{Ab}}(G, \mathbb{C}^\times)$ is
the \emph{Pontryagin dual} of a finite abelian group $G$. Now, if I
saw this written down with this notation (and I had known that
evaluation gives an isomorphism $\mathrm{ev} : G \to G^{\vee\vee}$,
which is not hard to see if one appeals to the structure theorem for
finite abelian groups), and Levent had said ``orthogonality
relations'' (observe $\mathrm{ev}(g)$ is an irreducible character on
$G^\vee$ for $g \in G$), the statement would be obvious to me. Indeed
from this perspective, i.e., in hindsight, the statement is utterly
trivial.

But getting there was far from trivial. True to my own form, I
floundered while Levent hinted me towards this final result. He
started by just asking me to write down an algebraic form for that
indicator function. To be fair, he hinted right away that this had to
do with representation theory, but I'm way too stupid to have foreseen
that. Indeed it took me \emph{way} too long (extrapolating from
$n=2,3$) to arrive at the expression
\[
p \mapsto \frac{1}{n} \sum_{k \in \mathbb{Z}/n} e^{2\pi ik(p-a)/n}.
\]
The problem with this is that Dirichlet's theorem really has to do
with $(\mathbb{Z}/n)^\times$ and $\varphi(n)$, not $\mathbb{Z}/n$ and
$n$. So these \emph{additive characters} (as I've learned to call them
in this setting) $x \mapsto e^{2\pi ikx/n}$ just won't do. What we
want is to find the proper analogy with \emph{multiplicative
  characters} $\chi \in ((\mathbb{Z}/n)^\times)^\vee$. This also took
me ridiculously long (after learning about Pontryagin duality),
expecting na\"ively that $\sum_{k \in \mathbb{Z}/n}$ would be replaced
by $\prod_\chi$. But anyway, with sufficient nudging (and profanity
and ``yelling at each other'', since this is how Levent does math
apparently...it works...) I got to the analogous
\[
p \mapsto \frac{1}{\varphi(n)} \sum_{\chi \in
  ((\mathbb{Z}/n)^\times)^\vee} \chi(p/a) = \frac{1}{\varphi(n)}
\sum_{\chi \in ((\mathbb{Z}/n)^\times)^\vee} \chi(p)\overline{\chi(a)}.
\]

That took a lot more work than looking at the answer and declaring it
trivial. Also it made me feel like an idiot (though we decided it
would be a lot worse when, in a few years, he screams obscenities at
his students in front of five hundred of their peers). But seriously,
the exercise without a doubt made me understand everything that
happened infinitely better, and especially forced me to think about
analogies and related questions. To paraphrase Frost, I was forced to
think inventively, and that made all the difference.

It's definitely a cliche that doing math is the only way to really
learn it. But I think it's easy to forget this when we get all this
abstraction thrown in our faces; I end up spending a lot of time
trying to absorb and understand, and this is fine, but today has
really convinced me that it's imperative to go back later and try to
\emph{reinvent} (with some help, probably...) the fundamentals. I
decided to call this \emph{learnventing}, for the obvious reason, and
to pay tribute to that troll Levent. So this summer I'm going to try
to do as much of this as possible, and hopefully this will turn into
some blogging. In particular I really want to learnvent some basic
algebra (representation theory, Galois theory) and some of the cool
consequences from all the machinery of basic algebraic topology.

This was my first real post. Congratulations, Arpon, you hero.
